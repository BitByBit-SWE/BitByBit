% contenuto_riunione.tex
% ----------------------
% EDITA solo questo file per ogni riunione.
% Sostituisci i valori fra {} e aggiungi/rimuovi righe nelle liste.
% NON inserire \documentclass o \begin{document} qui.

% ---------- METADATI (modifica solo qui) ----------
\newcommand{\nomegruppo}{\textbf{BitByBit}}
\newcommand{\numeroverbale}{01}
\newcommand{\datariunione}{20 Ottobre 2025}
\newcommand{\orainizio}{18}
\newcommand{\orafine}{19}
\newcommand{\durata}{1h}
\newcommand{\piattaforma}{Discord}
% --------------------------------------------------

% ---------- HEADER VISIBILE ---------- 
\begin{center}
    \vspace*{0.5em}
    \includegraphics[width=0.25\textwidth]{logo.png}\\[1em]
    {\LARGE \textbf{Verbale di Riunione n.\numeroverbale}}\\[0.7em]
    {\large Gruppo: \nomegruppo}\\[0.3em]
    {\normalsize Data: \textbf{\datariunione}}\\[0.3em]
    {\normalsize Ora inizio: \textbf{\orainizio} \quad Ora fine: \textbf{\orafine}}\\[0.3em]
    {\normalsize Durata: \textbf{\durata}}\\[0.3em]
    {\normalsize Piattaforma: \textbf{\piattaforma}}\\[1em]
\end{center}

\vspace{0.5em}
\hrule
\vspace{1em}

% ---------- PARTECIPANTI (modifica elenco) ----------
\section*{Partecipanti}
\begin{itemize}
    \item Dennis Parolin
    \item Riccardo Manisi
    \item Gabriele Scaggiante
    \item Ferdinando Fracasso
    \item Giovanni Visentin
    % Aggiungi o rimuovi \item qui sotto secondo necessità
\end{itemize}

% ---------- ASSENTI (modifica elenco) ----------
\section*{Assenti}
\begin{itemize}
    \item Nessuno
    % Se ci sono assenti, sostituisci la riga sopra con i nomi
\end{itemize}

% ---------- ORDINE DEL GIORNO (modifica elenco) ----------
\section*{Ordine del giorno}
\begin{enumerate}
    \item Creazione della repository condivisa
    \item Decisione degli appalti
    % Aggiungi o modifica punti qui
\end{enumerate}

% ---------- DISCUSSIONE E DECISIONI (modifica testo e lista) ----------
\section*{Discussione e decisioni}
Dopo la creazione della repository condivisa da tutti i componenti del gruppo, si è deciso di discutere su quale appalto potesse soddisfare la maggioranza deio componenti

\begin{itemize}
    \item È stato deciso di cercare di aggiudicarsi l'appalto C6
    \item Viene mandata una mail alla azienda relativa l'appalto scelto chiedendo di fissare un appuntamento per schiarire eventuali dubbi riguardanti le attività da svolgere
    % Aggiungi o modifica decisioni qui
\end{itemize}

% ---------- PROSSIME ATTIVITÀ (modifica elenco con nomi e scadenze) ----------
\section*{Prossime attività}
\begin{itemize}
    \item N.A
    % Aggiungi o modifica attività qui
\end{itemize}

