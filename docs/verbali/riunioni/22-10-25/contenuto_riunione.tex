TheVirtuousKhada
pyrovenyx
Invisibile

Questo è l'inizio del canale #file. 
Ferdinando

 — Ieri alle 18:26
Features:
Calcolo costi per ciascun ruolo
Calcolo costo totale
Verifica che le ore rientrino nei margini stabiliti
Verifica che il costo totale sia sopra il budget minimo
 
Tipo di allegato: spreadsheet
Calcolatore_preventivo.xlsx
9.86 KB
TheVirtuousKhada

 — Ieri alle 18:52
Tipo di allegato: acrobat
verbale2.pdf
232.05 KB
% contenuto_riunione.tex
% ----------------------
% EDITA solo questo file per ogni riunione.
% Sostituisci i valori fra {} e aggiungi/rimuovi righe nelle liste.
% NON inserire \documentclass o \begin{document} qui.

% ---------- METADATI (modifica solo qui) ----------
\newcommand{\nomegruppo}{\textbf{BitByBit}}
\newcommand{\numeroverbale}{02}
\newcommand{\datariunione}{22 Ottobre 2025}
\newcommand{\orainizio}{17.10}
\newcommand{\orafine}{18.45}
\newcommand{\durata}{1h 35min}
\newcommand{\piattaforma}{Discord}
% --------------------------------------------------

% ---------- HEADER VISIBILE ---------- 
\begin{center}
    \vspace*{0.5em}
    \includegraphics[width=0.25\textwidth]{logo.png}\\[1em]
    {\LARGE \textbf{Verbale di Riunione n.\numeroverbale}}\\[0.7em]
    {\large Gruppo: \nomegruppo}\\[0.3em]
    {\normalsize Data: \textbf{\datariunione}}\\[0.3em]
    {\normalsize Ora inizio: \textbf{\orainizio} \quad Ora fine: \textbf{\orafine}}\\[0.3em]
    {\normalsize Durata: \textbf{\durata}}\\[0.3em]
    {\normalsize Piattaforma: \textbf{\piattaforma}}\\[1em]
\end{center}

\vspace{0.5em}
\hrule
\vspace{1em}

% ---------- PARTECIPANTI (modifica elenco) ----------
\section*{Partecipanti}
\begin{itemize}
    \item Dennis Parolin
    \item Riccardo Manisi
    \item Ferdinando Fracasso
    \item Giovanni Visentin
    % Aggiungi o rimuovi \item qui sotto secondo necessità
\end{itemize}

% ---------- ASSENTI (modifica elenco) ----------
\section*{Assenti}
\begin{itemize}
    \item Gabriele Scaggiante
    % Se ci sono assenti, sostituisci la riga sopra con i nomi
\end{itemize}

% ---------- ORDINE DEL GIORNO (modifica elenco) ----------
\section*{Ordine del giorno}
\begin{enumerate}
    \item Stilare un "to-do" 
    \item Discussione sul way-of-working
    \item Aggiornamento repository github
    % Aggiungi o modifica punti qui
\end{enumerate}

% ---------- DISCUSSIONE E DECISIONI (modifica testo e lista) ----------
\section*{Discussione e decisioni}
Inizialmente, è stato stilato un elenco di attività da svolgere per chiarire quali siano le macro attività. Inoltre, abbiamo discusso per capire come è possibile suddividersi le ore di lavoro. Successivamente, è iniziata una discussione per creare e organizzare uno scheletro di way-of-working.
Dopodiché, sono state aggiunte issue ed è stata creata una milestone per la repository.
Infine abbiamo confermato il meeting con l'azienda dell'appalto C6 che verrà fatto in data \textbf{23/10/25} alle \textbf{15.30}

% ---------- PROSSIME ATTIVITÀ (modifica elenco con nomi e scadenze) ----------
\section*{Prossime attività}
\begin{itemize}
    \item Migliorare ulteriormente la repository di github (aggiungere automatismi)  — scadenza \textbf{26/10}.
    \item Creare primo diario di bordo  — scadenza \textbf{27/10}.
    % Aggiungi o modifica attività qui
\end{itemize}
% ---------- REDAZIONE E REVISIONE ----------
\vspace{2em}
\hrule
\vspace{0.8em}
\section*{Redazione e revisioni del documento}

\begin{center}
\renewcommand{\arraystretch}{1.3} % aumenta leggermente l'altezza delle righe
\begin{tabular}{|p{0.18\linewidth}|p{0.22\linewidth}|p{0.20\linewidth}|p{0.28\linewidth}|}
\hline
\textbf{Ruolo} & \textbf{Nome} & \textbf{Data e ora} & \textbf{Descrizione} \\
\hline
Redatto da & Dennis Parolin & 22/10/2025 – 18:45 & Stesura iniziale del verbale \\
\hline
Revisionato da & Riccardo Manisi & 22/10/25 - 18.45 & Controllo e conferma del verbale \\
\hline
Modifiche &  &  &  \\
\hline
\end{tabular}
\end{center}
Riduci
contenuto_riunione.tex
4 KB

% contenuto_riunione.tex
% ----------------------
% EDITA solo questo file per ogni riunione.
% Sostituisci i valori fra {} e aggiungi/rimuovi righe nelle liste.
% NON inserire \documentclass o \begin{document} qui.

% ---------- METADATI (modifica solo qui) ----------
\newcommand{\nomegruppo}{\textbf{BitByBit}}
\newcommand{\numeroverbale}{02}
\newcommand{\datariunione}{22 Ottobre 2025}
\newcommand{\orainizio}{17.10}
\newcommand{\orafine}{18.45}
\newcommand{\durata}{1h 35min}
\newcommand{\piattaforma}{Discord}
% --------------------------------------------------

% ---------- HEADER VISIBILE ---------- 
\begin{center}
    \vspace*{0.5em}
    \includegraphics[width=0.25\textwidth]{logo.png}\\[1em]
    {\LARGE \textbf{Verbale di Riunione n.\numeroverbale}}\\[0.7em]
    {\large Gruppo: \nomegruppo}\\[0.3em]
    {\normalsize Data: \textbf{\datariunione}}\\[0.3em]
    {\normalsize Ora inizio: \textbf{\orainizio} \quad Ora fine: \textbf{\orafine}}\\[0.3em]
    {\normalsize Durata: \textbf{\durata}}\\[0.3em]
    {\normalsize Piattaforma: \textbf{\piattaforma}}\\[1em]
\end{center}

\vspace{0.5em}
\hrule
\vspace{1em}

% ---------- PARTECIPANTI (modifica elenco) ----------
\section*{Partecipanti}
\begin{itemize}
    \item Dennis Parolin
    \item Riccardo Manisi
    \item Ferdinando Fracasso
    \item Giovanni Visentin
    % Aggiungi o rimuovi \item qui sotto secondo necessità
\end{itemize}

% ---------- ASSENTI (modifica elenco) ----------
\section*{Assenti}
\begin{itemize}
    \item Gabriele Scaggiante
    % Se ci sono assenti, sostituisci la riga sopra con i nomi
\end{itemize}

% ---------- ORDINE DEL GIORNO (modifica elenco) ----------
\section*{Ordine del giorno}
\begin{enumerate}
    \item Stilare un "to-do" 
    \item Discussione sul way-of-working
    \item Aggiornamento repository github
    % Aggiungi o modifica punti qui
\end{enumerate}

% ---------- DISCUSSIONE E DECISIONI (modifica testo e lista) ----------
\section*{Discussione e decisioni}
Inizialmente, è stato stilato un elenco di attività da svolgere per chiarire quali siano le macro attività. Inoltre, abbiamo discusso per capire come è possibile suddividersi le ore di lavoro. Successivamente, è iniziata una discussione per creare e organizzare uno scheletro di way-of-working.
Dopodiché, sono state aggiunte issue ed è stata creata una milestone per la repository.
Infine abbiamo confermato il meeting con l'azienda dell'appalto C6 che verrà fatto in data \textbf{23/10/25} alle \textbf{15.30}

% ---------- PROSSIME ATTIVITÀ (modifica elenco con nomi e scadenze) ----------
\section*{Prossime attività}
\begin{itemize}
    \item Migliorare ulteriormente la repository di github (aggiungere automatismi)  — scadenza \textbf{26/10}.
    \item Creare primo diario di bordo  — scadenza \textbf{27/10}.
    % Aggiungi o modifica attività qui
\end{itemize}
% ---------- REDAZIONE E REVISIONE ----------
\vspace{2em}
\hrule
\vspace{0.8em}
\section*{Redazione e revisioni del documento}

\begin{center}
\renewcommand{\arraystretch}{1.3} % aumenta leggermente l'altezza delle righe
\begin{tabular}{|p{0.18\linewidth}|p{0.22\linewidth}|p{0.20\linewidth}|p{0.28\linewidth}|}
\hline
\textbf{Ruolo} & \textbf{Nome} & \textbf{Data e ora} & \textbf{Descrizione} \\
\hline
Redatto da & Dennis Parolin & 22/10/2025 – 18:45 & Stesura iniziale del verbale \\
\hline
Revisionato da & Riccardo Manisi & 22/10/25 - 18.45 & Controllo e conferma del verbale \\
\hline
Modifiche &  &  &  \\
\hline
\end{tabular}
\end{center}
contenuto_riunione.tex
4 KB
