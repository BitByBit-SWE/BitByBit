% contenuto_riunione.tex
% ----------------------
% EDITA solo questo file per ogni riunione.
% Sostituisci i valori fra {} e aggiungi/rimuovi righe nelle liste.
% NON inserire \documentclass o \begin{document} qui.

% ---------- METADATI (modifica solo qui) ----------
\newcommand{\nomegruppo}{\textbf{BitByBit}}
\newcommand{\numeroverbale}{}
\newcommand{\datariunione}{}
\newcommand{\orainizio}{}
\newcommand{\orafine}{}
\newcommand{\durata}{}
\newcommand{\piattaforma}{}
% --------------------------------------------------

% ---------- HEADER VISIBILE ---------- 
\begin{center}
    \vspace*{0.5em}
    \includegraphics[width=0.25\textwidth]{logo.png}\\[1em]
    {\LARGE \textbf{Verbale di Riunione n.\numeroverbale}}\\[0.7em]
    {\large Gruppo: \nomegruppo}\\[0.3em]
    {\normalsize Data: \textbf{\datariunione}}\\[0.3em]
    {\normalsize Ora inizio: \textbf{\orainizio} \quad Ora fine: \textbf{\orafine}}\\[0.3em]
    {\normalsize Durata: \textbf{\durata}}\\[0.3em]
    {\normalsize Piattaforma: \textbf{\piattaforma}}\\[1em]
\end{center}

\vspace{0.5em}
\hrule
\vspace{1em}

% ---------- PARTECIPANTI (modifica elenco) ----------
\section*{Partecipanti}
\begin{itemize}
    \item Dennis Parolin
    \item Riccardo Manisi
    \item Gabriele Scaggiante
    \item Ferdinando Fracasso
    \item Giovanni Visentin
    % Aggiungi o rimuovi \item qui sotto secondo necessità
\end{itemize}

% ---------- ASSENTI (modifica elenco) ----------
\section*{Assenti}
\begin{itemize}
    \item Nessuno
    % Se ci sono assenti, sostituisci la riga sopra con i nomi
\end{itemize}

% ---------- ORDINE DEL GIORNO (modifica elenco) ----------
\section*{Ordine del giorno}
\begin{enumerate}
    \item Pianificazione attività settimanali
    \item Scelta architettura software
    \item Suddivisione dei compiti
    % Aggiungi o modifica punti qui
\end{enumerate}

% ---------- DISCUSSIONE E DECISIONI (modifica testo e lista) ----------
\section*{Discussione e decisioni}
Qui puoi mettere un breve riassunto generale della discussione (1-3 frasi),
poi elencare le decisioni concrete:

\begin{itemize}
    \item È stato deciso di utilizzare l'architettura \textbf{MVC}.
    \item Il front-end sarà sviluppato in \textit{React}, il back-end in \textit{Flask}.
    \item Mario si occuperà di preparare la documentazione tecnica entro il \textbf{25/10}.
    % Aggiungi o modifica decisioni qui
\end{itemize}

% ---------- PROSSIME ATTIVITÀ (modifica elenco con nomi e scadenze) ----------
\section*{Prossime attività}
\begin{itemize}
    \item \textbf{Anna Bianchi}: prototipo dell'interfaccia — scadenza \textbf{27/10}.
    \item \textbf{Luca Verdi}: setup del server e ambiente di test — scadenza \textbf{26/10}.
    \item \textbf{Mario Rossi}: documentazione tecnica — scadenza \textbf{25/10}.
    % Aggiungi o modifica attività qui
\end{itemize}

